\documentclass[11pt,reqno]{amsart}
\usepackage{mathptmx}
\usepackage{mathtools}
\usepackage{mathrsfs}
\usepackage[all]{xy}
\usepackage{stmaryrd}
\usepackage{fancyhdr}

\usepackage{pgf, tikz}
\usetikzlibrary{arrows, automata}

\usepackage{tikz-cd}
\usepackage{cite}
\usepackage{amsmath,amsfonts}
\usepackage{graphicx}
\usepackage{wrapfig}
\usepackage{array}
\usepackage{amsmath,amsthm,amssymb,hyperref}
\usepackage{exercise}

\usepackage{enumerate}
\usepackage{tikz}
\pagestyle{plain}
\usepackage[left=1.2in, right=1.2in, top=1in, bottom=1in]{geometry}
\usepackage{etoolbox}
\usepackage{color}
\usepackage{xcolor}
\patchcmd{\section}{\scshape}{\bfseries}{}{}
\makeatletter
\renewcommand{\@secnumfont}{\bfseries}
\makeatother
\xyoption{all}
\thispagestyle{empty}

\DeclareMathOperator{\Hom}{Hom}
\DeclareMathOperator{\Spec}{Spec}
\DeclareMathOperator{\Spv}{Spv}
\DeclareMathOperator{\Frac}{Frac}
\DeclareMathOperator{\Img}{Img}
\DeclareMathOperator{\Ker}{Ker}
\DeclareMathOperator{\Pic}{Pic}
\DeclareMathOperator{\Div}{Div}
\DeclareMathOperator{\dv}{Div}
\DeclareMathOperator{\Deg}{deg}
\DeclareMathOperator{\Aut}{Aut}
\DeclareMathOperator{\CSpec}{Cspec}
\newcommand{\angles}[1]{\langle #1 \rangle}
\newcommand{\Jac}{\textrm{Jac}}{}
\newcommand{\R}{\mathbb{R}}
\newcommand{\Z}{\mathbb{Z}}
\newcommand{\N}{\mathbb{N}}
\newcommand{\Q}{\mathbb{Q}}
\newcommand{\I}{\mathbf{I}}

\input xy
\xyoption{all}
\thispagestyle{empty}


%\usepackage{secdot}

\theoremstyle{definition}
\newtheorem{mydef}{\textbf{Definition}}[section]
\newtheorem{myeg}[mydef]{\textbf{Example}}
\newtheorem{conj}[mydef]{\textbf{Conjecture}}
\newtheorem*{noconj}{\textbf{Conjecture}}
\newtheorem{observ}[mydef]{\textbf{Observation}}
\newtheorem{question}[mydef]{\textbf{Question}}
\newtheorem{rmk}[mydef]{\textbf{Remark}}
\newtheorem*{que}{\textbf{Question}}
\newtheorem*{goal}{\textbf{Goal}}

\theoremstyle{plain}
\newtheorem{mythm}[mydef]{\textbf{Theorem}}
\newtheorem*{nothm}{\textbf{Theorem}}
\newtheorem*{nomainthm}{\textbf{Main Theorem}}
\newtheorem*{nothma}{\textbf{Theorem A}}
\newtheorem*{nothmb}{\textbf{Theorem B}}
\newtheorem*{nothmc}{\textbf{Theorem C}}
\newtheorem*{nothmd}{\textbf{Theorem D}}
\newtheorem*{nothme}{\textbf{Theorem E}}
\newtheorem*{nothmf}{\textbf{Theorem F}}
\newtheorem*{nothmg}{\textbf{Theorem G}}
\newtheorem{mytheorem}[mydef]{\textbf{Theorem}}
\newtheorem{lem}[mydef]{\textbf{Lemma}}
\newtheorem{pro}[mydef]{\textbf{Proposition}}
\newtheorem{claim}[mydef]{\textbf{Claim}}
\newtheorem{cor}[mydef]{\textbf{Corollary}}
\newtheorem{con}[mydef]{\textbf{Construction}}


\patchcmd{\abstract}{\scshape\abstractname}{\normalsize{\textbf{\abstractname}}}{}{}
\begin{document}


\title{On Picard Groups and Jacobians of Directed Graphs}

\author{Jaiung Jun}
\address{Department of Mathematics, State University of New York at New Paltz, NY 12561, USA}
\email{junj@newpaltz.edu}

\author{Matthew Pisano}
\address{Department of Mathematics, State University of New York at New Paltz, NY 12561, USA}
\email{pisanom1@newpaltz.edu}

\makeatletter
\@namedef{subjclassname@2020}{
	\textup{2020} Mathematics Subject Classification}
\makeatother

\subjclass[2020]{05C50, 05C76}
\keywords{Jacobian of a graph, sandpile group, critical group, chip-firing game, gluing graphs, cycle graph, Tutte
	polynomial, Tutte's rotor construction}

\begin{abstract}
	We explore a combinatorial game on finite graphs, called Chip-Firing Games,
	which has various connections to other areas, such as algebraic geometry, number theory and economics.
	To play the game, one first puts an integer amount of chips at each vertex. Then,
	each vertex is allowed to borrow or lend chips from its neighbors equally as the game progresses.
	One can study chip-firing games on a graph $G$ through a finitely generated abelian group
	$\textrm{Pic}(G)$ (Picard group) and its torsion subgroup $\textrm{Jac}(G)$ (Jacobian) which can
	be computed by using the Laplacian matrix of $G$.

	When a graph $G$ is directed, one may define $\textrm{Pic}(G)$ and $\textrm{Jac}(G)$ as in the case of undirected
	graphs by using Laplacian matrices, but computations become much more complicated in this case. For example,
	$\textrm{Pic}(G)=\mathbb{Z}$, when $G$ is a tree, from the matrix-tree theorem, which tells us that $|\textrm{Jac}(G)|$ is
	the number of spanning trees of $G$ in the undirected case.
	In the case of directed trees, even the rank of $\textrm{Pic}(T)$ can be arbitrarily large.
	For example, for any natural number $n$ we can construct a tree $T_n$ such that the rank of $\textrm{Pic}(T_n)$ is $n$.

	In our ongoing project, we study Picard groups and Jacobians for directed trees, cycles, and pseudotrees.
	Even in these seemingly simple cases, we find some new phenomenon. For instance, for the undirected cycle $C_n$,
	$\textrm{Jac}(C_n)=\mathbb{Z}_n$, however, we prove that in the directed case, for any given $m \leq n$, one can
	always find an orientation of $C_n$ in such a way that $\textrm{Jac}(C_n)$ is $\mathbb{Z}_m$.
	By closely examining trees and cycles, and how Picard groups and Jacobians change with (suitably defined)
	vertex and edge gluing, we obtain several results for pseudotrees.
\end{abstract}

\maketitle

\end{document}