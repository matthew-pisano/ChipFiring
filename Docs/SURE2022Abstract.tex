\documentclass[11pt,reqno]{amsart}
\usepackage{mathptmx}
\usepackage{mathtools}
\usepackage{mathrsfs}
\usepackage[all]{xy}
\usepackage{stmaryrd}
\usepackage{fancyhdr}

\usepackage{pgf, tikz}
\usetikzlibrary{arrows, automata}

\usepackage{tikz-cd}
\usepackage{cite}
\usepackage{amsmath,amsfonts}
\usepackage{graphicx}
\usepackage{wrapfig}
\usepackage{array}
\usepackage{amsmath,amsthm,amssymb,hyperref}
\usepackage{exercise}

\usepackage{enumerate}
\usepackage{tikz}
\pagestyle{plain}
\usepackage[left=1.2in, right=1.2in, top=1in, bottom=1in]{geometry}
\usepackage{etoolbox}
\usepackage{color}
\usepackage{xcolor}
\patchcmd{\section}{\scshape}{\bfseries}{}{}
\makeatletter
\renewcommand{\@secnumfont}{\bfseries}
\makeatother
\xyoption{all}
\thispagestyle{empty}

\DeclareMathOperator{\Hom}{Hom}
\DeclareMathOperator{\Spec}{Spec}
\DeclareMathOperator{\Spv}{Spv}
\DeclareMathOperator{\Frac}{Frac}
\DeclareMathOperator{\Img}{Img}
\DeclareMathOperator{\Ker}{Ker}
\DeclareMathOperator{\Pic}{Pic}
\DeclareMathOperator{\Div}{Div}
\DeclareMathOperator{\dv}{Div}
\DeclareMathOperator{\Deg}{deg}
\DeclareMathOperator{\Aut}{Aut}
\DeclareMathOperator{\CSpec}{Cspec}
\newcommand{\angles}[1]{\langle #1 \rangle}
\newcommand{\Jac}{\textrm{Jac}}{}
\newcommand{\R}{\mathbb{R}}
\newcommand{\Z}{\mathbb{Z}}
\newcommand{\N}{\mathbb{N}}
\newcommand{\Q}{\mathbb{Q}}
\newcommand{\I}{\mathbf{I}}

\input xy
\xyoption{all}
\thispagestyle{empty}


%\usepackage{secdot}

\theoremstyle{definition}
\newtheorem{mydef}{\textbf{Definition}}[section]
\newtheorem{myeg}[mydef]{\textbf{Example}}
\newtheorem{conj}[mydef]{\textbf{Conjecture}}
\newtheorem*{noconj}{\textbf{Conjecture}}
\newtheorem{observ}[mydef]{\textbf{Observation}}
\newtheorem{question}[mydef]{\textbf{Question}}
\newtheorem{rmk}[mydef]{\textbf{Remark}}
\newtheorem*{que}{\textbf{Question}}
\newtheorem*{goal}{\textbf{Goal}}

\theoremstyle{plain}
\newtheorem{mythm}[mydef]{\textbf{Theorem}}
\newtheorem*{nothm}{\textbf{Theorem}}
\newtheorem*{nomainthm}{\textbf{Main Theorem}}
\newtheorem*{nothma}{\textbf{Theorem A}}
\newtheorem*{nothmb}{\textbf{Theorem B}}
\newtheorem*{nothmc}{\textbf{Theorem C}}
\newtheorem*{nothmd}{\textbf{Theorem D}}
\newtheorem*{nothme}{\textbf{Theorem E}}
\newtheorem*{nothmf}{\textbf{Theorem F}}
\newtheorem*{nothmg}{\textbf{Theorem G}}
\newtheorem{mytheorem}[mydef]{\textbf{Theorem}}
\newtheorem{lem}[mydef]{\textbf{Lemma}}
\newtheorem{pro}[mydef]{\textbf{Proposition}}
\newtheorem{claim}[mydef]{\textbf{Claim}}
\newtheorem{cor}[mydef]{\textbf{Corollary}}
\newtheorem{con}[mydef]{\textbf{Construction}}


\patchcmd{\abstract}{\scshape\abstractname}{\normalsize{\textbf{\abstractname}}}{}{}
\begin{document}


\title{On Picard groups of directed graphs}
%

\author{Jaiung Jun}
\address{Department of Mathematics, State University of New York at New Paltz, NY 12561, USA}
\email{junj@newpaltz.edu}

\author{Matthew Pisano}
\address{Department of Mathematics, State University of New York at New Paltz, NY 12561, USA}
\email{pisanom1@newpaltz.edu}


%

%
\makeatletter
\@namedef{subjclassname@2020}{%
	\textup{2020} Mathematics Subject Classification}
\makeatother

\subjclass[2020]{05C50, 05C76}
\keywords{Jacobian of a graph, sandpile group, critical group, chip-firing game, gluing graphs, cycle graph, Tutte
	polynomial, Tutte's rotor construction}

\begin{abstract}
	We explore a combinatorial game on finite graphs, called Chip-Firing Games,
	which has various connections to other areas, such as algebraic geometry, number theory and economics.
	Roughly speaking, to play the game, one first puts an integer amount of chips at each vertex. Then,
	each vertex is allowed to borrow or lend chips from all of its neighbors equally the game progresses.

	A configuration of chips on the vertices of a graph is called a divisor\textcolor{red}{, a vector in
	$\mathbb{Z}^n$ for a graph of size $n$}. The degree of divisor is the total number of chips combined on the vertices.
	The collection of all divisors on a graph defines a free abelian group $\textrm{Div}(G)$, the divisor
	group of $G$. Under the equivalence relation $\sim$ generated by borrowing and lending moves, one obtain
	the Picard group $\Pic(G)=\Div(G)/\sim$. The Jacobian $\Jac(G)$ of $G$ is the subgroup of $\Pic(G)$
	consisting of the degree-zero divisors. These groups can be computed from the Laplacian matrix of a graph $G$.
	\textcolor{red}{Through the structure of these groups, we gain a better understanding of how any given game
	can be played and evolve.}

	When a graph $G$ is directed, one may define $\Pic(G)$ and $\Jac(G)$ as in the case of undirected
	graphs by using Laplacian matrices, but computations become much more complicated in this case. For example,
	$\Pic(G)=\mathbb{Z}$, when $G$ is a tree, from the matrix-tree theorem, which tells us that $|\Jac(G)|$ is
	the number of spanning trees of $G$ in the undirected case.  For the case of directed trees $T$, even the
	rank of $\Pic(T)$ can be arbitrarily large depending on the size of $T$.

	In our ongoing project, we study Picard groups and Jacobians for trees, cycles, and pseudotrees.
	Even in these seemingly simple cases, we find some new phenomenon. For instance, for the undirected cycle $C_n$,
	$\Jac(C_n)=\mathbb{Z}_n$, however, we prove that in the directed case, for any given $m \leq n$, one can
	always find an orientation of $C_n$ in such a way that $\Jac(C_n)$ (with this orientation) is $\mathbb{Z}_m$.
	By closely examining trees and cycles, and how Picard groups and Jacobians change with (suitable defined)
	vertex and edge gluing, we obtain several results for pseudotrees.
\end{abstract}

\maketitle

\end{document}