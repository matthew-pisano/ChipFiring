\section{Propositions}
	\subsection{Calculating Sinks and Sources}
		Vertices on a directed graph can be classified in three ways, as a sink, source, or neither depending on
		the direction of the edges connected to it.  A sink is a vertex where all edges are directed into that vertex,
		a source is a vertex where all edges are directed away from that vertex and a vertex is neither
		when it has a mixture of the two.  The number of sinks
		and sources can be calculated inductively by reconstructing the original graph from a single vertex.

		Beginning with any two vertices that are adjacent on the original graph, observe their connecting edge.
		If the edge is directed, the count of sinks and sources begins at 1 for both.  If the edge
		is undirected (bidirectional), the count begins at zero. Next, pick a vertex adjacent to one of the original
		edges and observe its connection.  From this point on, when a new edge is observed the number of sinks and
		sources changes on a series of rules.  For these rules, let the vertex being added to be $V_1$ and
		the vertex being added be $V_2$.
		\begin{enumerate}
			\item If $V_1$ is neither a sink nor a source.
			\begin{enumerate}
				\item Adding an edge directed towards $V_2$ adds a sink.
				\item Adding an edge directed away from $V_2$ adds a source.
				\item Adding a bidirectional edge between $V_1$ and $V_2$ adds nothing.
			\end{enumerate}
			\item If $V_1$ is a source.
			\begin{enumerate}
				\item Adding an edge directed towards $V_2$ adds a sink.
				\item Adding an edge directed away from $V_2$ adds nothing (Here the number of sources stays
					\label{itm:noChange}
					the same as $V_1$ is no longer a source but $V_2$ now is).
				\item Adding a bidirectional edge between $V_1$ and $V_2$ removes a source.
			\end{enumerate}
			\item If $V_1$ is a sink.
			\begin{enumerate}
				\item Adding an edge directed towards $V_2$ adds nothing (See~\ref{itm:noChange}).
				\item Adding an edge directed away from $V_2$ adds a source.
				\item Adding a bidirectional edge between $V_1$ and $V_2$ removes a sink.
			\end{enumerate}
		\end{enumerate}
		When accounting for a vertex that has multiple connections, apply these rules for every $V_{1i}$ that $V_2$
		is connected to, taking into account all the edges of $V_2$ when evaluating its class.  After every vertex has
		been accounted for, the number of sinks and sources will have been calculated in polynomial time between $O(n)$
		at the average case and $O(n^2)$ for the case of a connected graph.

	%\subsection{Using Sinks and Sources to Calculate $Rank(\Pic(G))$}
	%	$\Pic(G)$ is often in the form $\mathbb{Z}_1 \times \dots \times \mathbb{Z}_n \times \mathbb{Z}^m$ where $m$ is
	%	the rank of the picard group.  When calculating $\Pic(G)$, its $Rank()$ is closely correlated with the number
	%	of sinks and sources of the graph.  $Rank(\Pic(G))$ appears to always either
	%	in $range(sources, sinks)$ (inclusive) or one greater than $\max(sources, sinks)$.

	%~\cite[Corollary 3.5]{wagner2000critical}